\chapter{分支语句和逻辑运算符}
\section{字符函数库 cctype}
C++从 C 语言继承了一个与字符相关的、非常方便的函数软件包,它可以简化诸如确定字符是否为大写字母、数字、标点符号等工作,这些函数的原型是在头文件 cctype(老式的风格中为 ctype.h)中定义的。
\begin{table}[h]
	\centering
	\caption{cctype 中的字符函数}
	\begin{tabular}{c|l}
		\hline
		函数名称 & 返回值 \\ \hline
		isalnum() & 如果参数是字母数字,即字母或数字,该函数返回 true \\ \hline
		isalpha() & 如果参数是字母,该函数返回 true \\ \hline
		iscntrl() & 如果参数是控制字符,该函数返回 true \\ \hline
		isdigit() & 如果参数是数字($0 \sim 9$),该函数返回 true \\ \hline
		isgraph() & 如果参数是除空格之外的打印字符,该函数返回 true \\ \hline
		islower() & 如果参数是小写字母,该函数返回 true \\ \hline
		isprint() & 如果参数是打印字符(包括空格),该函数返回 true \\ \hline
		ispunct() & 如果参数是标点符号,该函数返回 true \\ \hline
		\multirow{2}*{isspace()} & 如果参数是标准空白字符,如空格、进纸、换行符、回车、水平制 \\  &表符或者垂直制表符,该函数返回 true \\ \hline
		isupper() & 如果参数是大写字母,该函数返回 true \\ \hline
		isxdigit() & 如果参数是十六进制数字,该函数返回 true \\ \hline
		tolower() & 如果参数是大写字符,则返回其小写,否则返回该参数 \\ \hline
		toupper() & 如果参数是小写字符,则返回其大写,否则返回该参数 \\ \hline
	\end{tabular}
\end{table}
\section{?: 运算符}
?: 运算符也被称为条件运算符,它是 C++ 中唯一一个需要 3 个操作数的运算符。通用格式为
\begin{tcblisting}{listing engine=minted,minted style=trac,minted language=c++,minted options={linenos,numbersep=5pt},colback=red!5!white,colframe=red!75!black,listing only}
	expression1 ? expression2 : expression3
	5 > 3 ? 10 : 12 //5>3 is true, so expression value is 10
\end{tcblisting}
\section{简单文件输入/输出}
\subsection{写入到文本文件中}
对于文件输入,C++ 使用了类似于 cout 的东西。下面列出有关将 cout 用于控制台输出的基本事实
\begin{itemize}
	\setlength{\itemsep}{0pt}
	\setlength{\parsep}{0pt}
	\setlength{\parskip}{0pt}
	\item 必须包含头文件 iostream。
	\item 头文件 iostream 定义了一个处理输出的 ostream 类。
	\item 头文件 iostream 声明了一个名为 cout 的 ostream 对象。
	\item 必须指明名称空间 std;例如,为引用元素 cout 和 endl,必须使用编译指令 using 或前缀 std::。
	\item 可以结合使用 cout 和运算符<<来显示各种类型的数据。
\end{itemize}
文件输出与此极为相似。
\begin{itemize}
	\setlength{\itemsep}{0pt}
	\setlength{\parsep}{0pt}
	\setlength{\parskip}{0pt}
	\item 必须包含头文件 fstream。
	\item 头文件 fstream 定义了一个用于处理输出的 ofstream 类。
	\item 需要声明一个或多个 ofstream 变量(对象),并以自己喜欢的方式对其进行命名,条件是遵守常用的命名规则。
	\item 必须指明名称空间 std;例如,为引用元素 ofstream,必须使用编译指令 using 或前缀 std::。
	\item 需要将 ofstream 对象与文件关联起来。为此,方法之一是使用 open() 方法。
	\item 使用完文件后,应使用 close() 将其关闭。
	\item 可结合使用 ofstream 对象和运算符<<来输出各种类型的数据。
\end{itemize}
\begin{tcblisting}{listing engine=minted,minted style=trac,minted language=c++,minted options={linenos,numbersep=5pt},colback=red!5!white,colframe=red!75!black,listing only}
	ofstream outFile  //outFile an ofstream object
	ofstream fout  //fout an ofstream object
	outFile.open("fish.txt");
	char filename[50];
	cin >> filename;
	fout.open(filename);
	double wt = 125.8;
	outFile << wt; //write a number to fish.txt
	char line[81] = "Object are closer than they appear.";
	fout << line << endl;
\end{tcblisting}
声明一个 ofstream 对象并将其同文件关联起来后,便可以像使用 cout 那样使用它。所有可用于 cout 的操作和方法都可用于 ofstream 对象。

默认情况下,open() 将首先截断该文件,即将其长度截短到零——丢其原有的内容,然后将新的输出加入到该文件中。
\begin{marker}
	打开已有的文件,以接受输出时,默认将它其长度截短为零,因此原来的内容将丢失。
\end{marker}
\subsection{读取文本文件}
回顾文本文件输入,它是基于控制台输入的。控制台输入涉及多个方面,下面首先总结这些方面。
\begin{itemize}
	\setlength{\itemsep}{0pt}
	\setlength{\parsep}{0pt}
	\setlength{\parskip}{0pt}
	\item 必须包含头文件 iostream。
	\item 头文件 iostream 定义了一个用处理输入的 istream 类。 
	\item 头文件 iostream 声明了一个名为 cin 的 istream 变量(对象)。
	\item 必须指明名称空间 std;例如,为引用元素 cin,必须使用编译指令 using 或前缀 std::。
	\item 可以结合使用 cin 和运算符>>来读取各种类型的数据。
	\item 可以使用 cin 和 get() 方法来读取一个字符,使用cin 和 getline() 来读取一行字符。
	\item 可以结合使用 cin 和 eof()、fail()方法来判断输入是否成功。
	\item 对象 cin 本身被用作测试条件时,如果最后一个读取操作成功,它将被转换为布尔值 true,否则被转换为 false。
\end{itemize}
文件输入与此极其相似:
\begin{itemize}
	\setlength{\itemsep}{0pt}
	\setlength{\parsep}{0pt}
	\setlength{\parskip}{0pt}
	\item 必须包含文件头 fstream。
	\item 头文件 fstream 定义了一个用于处理输入的 ifstream 类。
	\item 需要声明一个或多个 ifstream 变量(对象),并以自己喜欢的方式对其进行命名,条件是遵守常用的命名规则。
	\item 必须指明名称空间 std;例如,为引用元素 ifstream,必须使用编译指令 using 或前缀 std::。
	\item 需要将 ifstream 对象与文件关联起来。为此,方法之一是使用 open()方法。
	\item 使用完文件后,应使用 close() 方法将其关闭。
	\item 可结合使用 ifstream 对象和运算符>>来读取各种类型的数据。
	\item 可以使用 ifstream 对象和 get()方法来读取一个字符,使用 ifstream 对象和 getline()来读取一行字符。
	\item 可以结合使用 ifstream 和 eof()、fail()等方法来判断输入是否成功。
	\item  ifstream 对象本身被用作测试条件时,如果最后一个读取操作成功,它将被转换为布尔值 true,否则被转换为 false。
\end{itemize}
\begin{tcblisting}{listing engine=minted,minted style=trac,minted language=c++,minted options={linenos,numbersep=5pt},colback=red!5!white,colframe=red!75!black,listing only}
	ifstream inFile; // inFile an ifstream object
	ifstream fin; //fin an ifstream object
	inFile.open("bowling.txt");
	char filename[50];
	cin >> filename;
	fin.open(filename);
	double wt;
	inFile >> wt; //read a number from bowling.txt
	char line[81];
	fin.getline(line, 81); //read a line of text
\end{tcblisting}
{\color{red}声明一个 ifstream 对象并将其同文件关联起来后,便可以像使用 cin 那样使用它。所有可用于 cin 的操作和方法都可用于 ifstream 对象}
\begin{marker}
	Windows 文本文件的每行都以回车字符和换尾符结尾;通常情况下,C++在读取文件时将这两个字符转换为换行符,并在写入文件时执行相反的转换。有些文本编辑器不会自动在最后一行末尾加上换行符。此时需要在最后按下回车键,然后再保存文件。
\end{marker}