\chapter{循环}
\section{for 循环}
for 循环为执行重复的操作提供了循环渐进的步骤。
\begin{enumerate}
	\item 设置初始值
	\item 执行测试,看看循环是否应当继续进行
	\item 执行循环操作
	\item 更新用于测试的值
\end{enumerate}
\begin{tcblisting}{listing engine=minted,minted style=trac,minted language=c++,minted options={linenos,numbersep=8pt},colback=red!5!white,colframe=red!75!black,listing only}
	for(initialization; test-expression; update-expression)
		body
\end{tcblisting}
\subsection{副作用和顺序点}
C++ 就递增运算符生效做了一些规定。首先,副作用(side effect)指的是在计算表达式时对某些东西(如存储在变量中的值)进行了修改;顺序点(sequence point)是程序执行过程中的一个点,在这里,进入下一步之前将确保对所有的副作用都进行了评估。在 C++ 中,语句的分号是一个顺序点,这意味着程序处理下一条语句之前,赋值运算符、递增运算符和递减运算符执行的所有修改都必须完成。任何完整的表达式末尾都是一个顺序点。
\begin{tcblisting}{listing engine=minted,minted style=trac,minted language=c++,minted options={linenos,numbersep=8pt},colback=red!5!white,colframe=red!75!black,listing only}
	while (guest++ < 10)
		cout << guest << endl;
\end{tcblisting}
这里表达式 guest++ < 10 是一个完整表达式,C++ 将确保副作用(将 guest 加 1)在程序进入 cout 之前完成。而下面的语句中
\begin{tcblisting}{listing engine=minted,minted style=trac,minted language=c++,minted options={linenos,numbersep=8pt},colback=red!5!white,colframe=red!75!black,listing only}
	y = (4 + x++) + (6 + x++);
\end{tcblisting}
表达式 4 + x++ 不是一个完整的表达式,因此,C++ 不保证 x 的值在计算子表达式 4 + x++ 后立刻增加 1。在这个例子中,整条赋值语句是一个完整的表达式,而分号标注了顺序点。C++ 没有规定是在计算每个子表达式之后将 x 的值递增,还是在整个表达式计算完毕后才将 x 的值递增,所以应避免使用。
\subsection{前缀格式和后缀格式}
选择前缀格式还是使用后缀格式可能对程序的行为没有影响(for 循环中的条件语句),但是执行速度可能有细微的差别。对于内置类型和当代的编译器而言,这可能不是什么问题。然而,C++ 允许您针对类定义这些运算符,在这种情况下,用户这样定义前缀函数:将值加 1,然后返回结果;但后缀版本首先复制一个副本,将其加 1,然后将复制的副本返回。因此,对于类而言,前缀版本的效率比后缀版本高。
\subsection{递增/递减运算符和指针}
可将递增运算符用于指针和基本变量。将递增运算符用于指针时,将把指针的值增加其值相的数据类型占用的字节数,这种规则适用于对指针递增和递减
\begin{tcblisting}{listing engine=minted,minted style=trac,minted language=c++,minted options={linenos,numbersep=8pt},colback=red!5!white,colframe=red!75!black,listing only}
	double arr[5] = {21.1, 11.2, 23.4, 23.1, 43.2};
	double* pt = arr;  //pt points to arr[0]
	++pt;             //pt points to arr[1]
\end{tcblisting}
也可以结合使用这些运算符和 * 运算符来修改指针指向的值。前缀递增、前缀递减和解除引用运算符的优先级相同,以从右到左的方式进行结合。后缀递增和后缀递减的优先级相同,但比前缀运算符的优先级高,这两个运算符以从左到右的方式进行结合
\begin{tcblisting}{listing engine=minted,minted style=trac,minted language=c++,minted options={linenos,numbersep=8pt},colback=red!5!white,colframe=red!75!black,listing only}
	double x = *++pt;  //increment pointer, pt points to arr[2]
	++*pt;      
//increment the pointed to value, change 23.4 to 24.4
	(*pt)++;   //increment pointed-to value
	x = *pt++;
//dereference original location, then increment pointer
\end{tcblisting}
后缀运算符++的优先级更高,这意味着将运算符用于 pt,而不是 *pt,因此对指针递增。然后后缀运算符意味着将对原来的地址(\&arr[2])而不是递增后的新地址解除引用,因此 *pt++ 的值为 arr[2],但该语句执行完毕后,pt 的值将为 arr[3] 的地址。
%\begin{table}[h]
%	\centering
%	\caption{组合赋值运算符}
%	\begin{tabular}{c|c}
%		\hline 
%		操作符 & 作用(L为左操作数,R为右操作数) \\ \hline
%		+= & 将 L+R 赋给 L \\ \hline
%		-+ & 将 L-R 赋给 L \\ \hline
%		*= & 将 L*R 赋给 L \\ \hline
%		/= & 将 L/R 赋给 L \\ \hline
%		\%= & 将 L\%R 赋给 L  \\ \hline
%	\end{tabular}
%\end{table}

复合语句有一种有趣的特性。如果在语句块中定义一个新的变量,则仅当程序执行该语句块中的语句时,该变量才存在。执行完该语句块后,变量将被释放,即此变量仅在该语句块中才是可用的

在使用语句块时,如果在外部语句块中声明一个变量,而外部语句块中也有一个这种名称的变量,在声明位置到内部语句块结束的范围之内,新变量将隐藏旧变量;然后旧变量再次可见。
\begin{tcblisting}{listing engine=minted,minted style=trac,minted language=c++,minted options={linenos,numbersep=5pt},colback=red!5!white,colframe=red!75!black,listing only}
#include<iostream>
int main()
{
	using std::cout;
	using std::endl;
	int x = 20;    //original x
	{
		cout << x << endl;  //use original x
		int x = 100;        // new x
		cout << x << endl;  // use new x
	}
	cout << x << endl;    // use original x
	return 0;
}
\end{tcblisting}
\section{while 循环}
while 循环没有初始化和更新部分的 for 循环,它只有测试条件和循环体:
\begin{tcblisting}{listing engine=minted,minted style=trac,minted language =c++,minted options={linenos,numbersep=5pt},colback=red!5!white,colframe=red!75!black,listing only}
	while (test-condition)
		body
\end{tcblisting} 
\subsection{类型别名}
C++ 为类型建立别名的方式有两种。一种是使用预处理器:
\begin{tcblisting}{listing engine=minted,minted style=trac,minted language=c++,minted options={linenos,numbersep=5pt},colback=red!5!white,colframe=red!75!black,listing only}
#define BYTE char //reprocessor replace BYTE with char
\end{tcblisting}
这样,预处理器将在编译程序时用 char 替换所有的 BYTE,从而使 BYTE 成为 char 的别名。

第二种方法是使用 C++(和 C)的关键字 typedef 来创建别名。例如,要将 byte 作为 char 的别名,可以这样做:
\begin{tcblisting}{listing engine=minted,minted style=trac,minted language=c++,minted options={linenos,numbersep=5pt},colback=red!5!white,colframe=red!75!black,listing only}
typedef char byte; //make byte an alias for char
//typedef typename aliasname;
\end{tcblisting}
要让 byte\_pointer 成为 char 指针的别名,可将 byte\_pointer 声明为 char 指针,然后在前面加上 typedef:
\begin{tcblisting}{listing engine=minted,minted style=trac,minted language=c++,minted options={linenos,numbersep=5pt},colback=red!5!white,colframe=red!75!black,listing only}
typedef char * byte_pointer; 
// or 
#define float_pointer float *
float_pointer pa, pb;
// preprocessor convert to
float * pa, pb; //pa a pointer, pb just a float
\end{tcblisting}
对于一系列变量的声明,使用 \#define 可能并不适用,如上所示的第 6 行,预处理器置换声明时对后面的变量不一致。typedef 方法不会出现这样的问题。它能够处理更复杂的类型别名。
\section{do while 循环}
\begin{tcblisting}{listing engine=minted,minted style=trac,minted language=c++,minted options={linenos,numbersep=5pt},colback=red!5!white,colframe=red!75!black,listing only}
	do
		body
	while (test-expression);
\end{tcblisting}
\section{基于范围的 for 循环(C++11)}
C++11 新增了一种循环:基于范围(range-based)的 for 循环。这简化了一种常见的循环任务:对数组(或容器类,如 vector 和 array)的每个元素执行相同的操作
\begin{tcblisting}{listing engine=minted,minted style=trac,minted language=c++,minted options={linenos,numbersep=5pt},colback=red!5!white,colframe=red!75!black,listing only}
	double price[5] = {4.55, 12.11, 23.12, 33.12, 3.11};
	for (double x : price)
		cout << x << std::endl;
\end{tcblisting}
要修改数组的元素,需要使用不同的循环变量语法:
\begin{tcblisting}{listing engine=minted,minted style=trac,minted language=c++,minted options={linenos,numbersep=5pt},colback=red!5!white,colframe=red!75!black,listing only}
	for (double &x : price)
		x = x * 0.8;
\end{tcblisting}
符号 \& 表明 x 是一个引用变量(我认为这是直接对数组的地址进行引用,从而 x 就是真实的数组元素,而不是复制的副本,这时对 x 的改变就是对地址中存储值的改变)。
\section{循环和文本输入}
cin 对象支持 3 种不同模式的单字符输入,其用户接口各不相同。
\subsection{使用原始的 cin 进行输入}
如果程序要使用循环来读取来自键盘的文本输入,必须要有办法知道何时停止读取。一种方法是选择某个特殊字符——有时被称为哨兵字符,将其作为停止标记。
\begin{tcblisting}{listing engine=minted,minted style=trac,minted language=c++,minted options={linenos,numbersep=5pt},colback=red!5!white,colframe=red!75!black,listing only}
	//...
	cin >> ch;
	while (ch != '#')
	{
		cout << ch;
		++count;
		cin >> ch;
	}
//input
see ken run#really fast
//output
seekenrun
\end{tcblisting}
程序输出时省略了空格,原因在 cin。读取 char 值时,与读取其他基本类型一样,cin 将忽略空格和换行符。因此,输入中的空格没有被回显,也没有被包括在计数内。

而且发送给 cin 的输入被缓冲。这表示只有用户按下回车键后,输入的内容才会被发送给程序。这就是在运行程序时,可以在 \# 后输入字符的原因。
\subsection{使用 cin.get(cahr) 进行补救}
cin 所属的 istream 类中包含一个能够满足这种要求的成员函数。具体来说,成员函数 cin.get(ch) 读取输入中的下一个字符(即使它是空格),并将其赋给变量。不过该输入仍被缓冲。
但是在 C 语言中,要修改变量的值,必须将变量{\color{red}地址}传递给函数。而调用 cin.get(ch) 传递的是 ch,而不是 \&ch。在 C 语言中,这样的代码无效。而在 C++ 中有效,只要函数将参数声明为引用即可。
\subsection{文件尾条件}
使用诸如 \# 等符号来表示输入结束很难令人满意,这样的符号可能就是合法输入的组成部分。如果输入来自文件,则可以使用一种功能强大的技术——检测文件尾(EOF)。C++输入工具和操作系统协同工作,来检测文件尾并将这种信息告知程序。

读取文件中的信息似乎同 cin 和键盘输入没什么关系,但其实存在两个相关的地方。首先,很多操作系统(包括 Unix、Linux 和 Windows 命令提示符模式)都支持重定向,允许用文件替换键盘输入。其次,很多操作系统都允许通过键盘来模拟文件尾条件。在 Unix 中,可以在行首按下 Ctrl+D 来实现;在 Windows 命令提示符模式下,可以在任意位置按 Ctrl+Z 和 Enter。总之,很多 PC 编程环境都将 Ctrl+Z 视为模拟的 EOF,但具体细节(必须在行首还是可以在任何位置,是否必须按下回车键等)各不相同。

检测到 EOF 后,cin 将两位(eofbit 和 failbit)都设置为 1。可以通过成员函数 eof() 来查看 eofbit 是否被设置;如果检测到 EOF,则 cin.eof() 将返回 bool 值 true,否则返回 false。同样,如果 failbit 被设置为 1,则 fail() 成员函数返回 true,否则返回 false。注意,eof() 和 fail() 方法报告最近读取的结果;也就是说,它们在事后报告,而不是预先报告。因此应将 cin.eof() 或 cin.fail() 测试放在读取后。