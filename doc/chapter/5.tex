\chapter{函数探幽}
\section{内联函数}
内联函数是 C++为提高程序运行速度所做的一项改进。常规函数和内联函数之间的主要区别不在于编写方式,而在于 C++编译器如何将它们组合到程序中。常规函数调用使程序跳到另一个地址,并在函数结束时返回。而内联函数的编译代码与其他程序代码“内联”起来了。编译器将使用相应的函数代码替换函数调用。对于内联代码,程序无需跳到另一个位置处执行代码,再跳回来。因此,内联函数的运行速度比常规函数稍快,但代价是需要占用更多内存。

应有选择地使用内联函数。如果执行函数代码的时间比处理函数调用机制的时间长,则节省的时间将只占整个过程的很小一部分。如果代码执行时间很短,则内联调用就可以节省非内联调用使用的大部分时间。不过由于这个过程相当快,尽管节省了该过程的大部分时间,但节省的时间绝对值并不大,除非该函数经常被调用。使用方式:
\begin{itemize}
	\setlength{\itemsep}{0pt}
	\setlength{\parsep}{0pt}
	\setlength{\parskip}{0pt}
	\item 在函数声明前加上关键字 inline;
	\item 在函数定义前加上关键字 inline。
\end{itemize}
通常的做法是省略原型,将整个定义(函数头和所有函数代码)放在本应提供原型的地方。
\begin{marker}
将函数作为内联函数时,编译器并不一定会满足这种要求。它可能认为该函数过大或注意到函数调用了自己(内联函数不能递归),因此不将其作为内联函数;而有些编译器没有启用或实现这种特性。
\end{marker}
内联函数和常规函数一样,也是按值来传递参数的。如果参数为表达式,则函数将传递表达式的值。这使得 C++的内联功能远胜过 C 语言的宏定义。
\begin{tcolorbox}[title=内联与宏,colback=yellow!5!white,center title]
inline 工具是 C++新增的特性。C 语言使用预处理其语句\#define 来提供宏——内联代码的原始实现。例如,下面是一个计算平方的宏:

\#define SQUARE(X) X*X

这并不是通过传递参数实现的,而是通过文本替换来实现的——X是“参数”的符号标记。
\begin{ccode}
a = SQUARE(5.0); //replaced by a = 5.0*5.0;
b = SQUARE(4.5+7.5);//replaced by b = 4.5+7.5*4.5+7.5;
d = SQUARE(c++);//replaced by d = c++ * c++;
\end{ccode}
上述示例只有第一个能正常工作。可通过使用括号进行改进:
\#define SQUARE(X) ((X)*(X))
这样仍存在这样的问题,即宏不按值传递。即新的定义中,SQUARE(c++) 仍将 c 递增两次。
\end{tcolorbox}
\section{引用变量}
引用是已定义的变量的别名(另一个名称)。引用变量的主要用途是用作函数的形参。通过将引用变量用作参数,函数将使用原始数据,而不是副本。
\subsection{创建引用变量}
C++给\&符号赋予了另一个含义,将其用来声明引用。例如
\begin{ccode}
int rats;
int & rodents = rats; //makes rodents an alias for rats
\end{ccode}
其中,\&不是地址运算符,而是类型标识符的一部分。像指针一样,int \&指的是指向 int 的引用。上述引用声明允许将 rats 和 rodents 互换——它们指向相同的值和内存单元。

{\color{red}引用和指针有一定的区别。必须在声明引用是将其初始化,而不能像指针那样,先声明,再赋值。引用更接近 const 指针,必须在创建时进行初始化,一旦与某个变量关联起来,就将一直效忠于它。}
\subsection{将引用用作函数参数}
引用经常被用作函数参数,使得函数中的变量名成为调用程序中的变量的别名。这种传递参数的方法称为按引用传递。按引用传递允许被调用的函数能够访问调用函数中的变量。按值传递导致被调用函数使用调用程序的值的拷贝。
\section{引用的属性和特别之处}
如果想让函数使用传递给它的信息,而不对这些信息进行修改,同时又想使用引用,则应使用常量引用,即使用 const:
\begin{ccode}
double refcube(const double &ra);
\end{ccode}

按值传递的函数,可使用多种类型的实参。但是传递引用的限制将比较严格。例如,refcube(x+3.0),在现代C++中,这是错误的,因此表达式并不是变量。而有些较老的编译器将发出警告。有些情况下允许这样做,此时程序将创建一个临时的无名变量,并将其初始化为表达式的值。

如果实参与引用参数不匹配,C++将生成临时变量。当前,仅当参数为 const 引用时,C++才允许这样做。如果引用参数是 const,则编译器将在下面情况下生成临时变量:
\newenvironment{itemizepro}{\begin{itemize}%
	\setlength{\itemsep}{0pt}%
	\setlength{\parsep}{0pt}%
	\setlength{\parskip}{0pt}%
	}{\end{itemize}}
\begin{itemizepro}
	\item 实参的类型正确,但不是左值
	\item 实参的类型不正确,但可以转换为正确的类型
\end{itemizepro}
{\color{red}左值参数是可被引用的数据对象,例如,变量、数组元素、结构成员、引用和解除引用的指针都是左值。非左值包括字面常量(\emph{用引号括起的字符串除外,它们由其地址表示})和包含多项的表达式。}在C中,左值最初指的是可出现在赋值语句左边的实体,但是这是引入关键字 const 之前的情况。现在,常规变量和 const 变量都可以视为左值,因为可以通过地址访问它们。
\begin{ccode}
double refcube(const double &ra);
double side = 3.0;
double * pd = &side;
double & rd = side;
long edge = 5L;
double len[4] = {2.0, 5.0, 10.0, 12.0};
double c1 = refcube(side);//ra is side
double c2 = refcube(lens[2]);//ra is lens[2]
double c3 = refcube(rd);//ra is rd
double c4 = refcube(*pd);//ra is *pd
\end{ccode}
\begin{ccode}
double c5 = refcube(edge);//ra is temporary variable
double c6 = refcube(7.0);//ra is temporary
double c7 = refcube(side+10.0);//ra is temporary
\end{ccode}
上述代码中将生成一个临时匿名变量,并让 ra 指向它。这些临时变量只在函数调用期间存在,此后编译器便可以随意将其删除。
\begin{marker}
尽量避免临时变量的创建
\end{marker}
\subsection{将引用用于结构}
引用非常适合用于结构和类。使用结构引用参数的方式与使用基本变量引用相同。

返回引用需要注意的一点是,应避免返回函数终止时不再存在的内存单元引用。如:
\begin{ccode}
const free_throws & clone(free_throws & ft)
{
	free_throws newguy;
	newguy = ft;
	retunr newguy;
}
\end{ccode}
该函数返回一个指向临时变量的引用,函数运行完毕后它将不再存在。同样,也应避免返回指向临时变量的指针。
\subsection{将引用用于类对象}
将类对象传递给函数时,C++通常的做法是使用引用。
\subsection{对象、继承和引用}
\section{函数重载}
默认参数使得能够使用不同数目的参数调用同一个函数,而函数多态(函数重载)使得能够使用多个同名的函数。可可以通过函数重载来设计一系列函数——它们完成相同的工作,但使用不同的参数列表。

函数重载的关键是函数的参数列表——也称函数特征标。如果两个函数的参数数目和类型相同,同时参数的排列顺序也相同,则它们的特征标相同,而变量名是无关紧要的。C++允许定义名称相同的函数,条件是它们的特征标不同。
\begin{ccode}
void print(const char * str, int width); //#1
void print(double d, int width);         //#2
void print(long l, int width);           //#3
void print(int i, int width);            //#4
void print(const char * str);            //#5
\end{ccode}
使用 print()函数时,编译器将根据所采取得用法使用相应特征标得原型:
\begin{ccode}
print("Panda", 6);    //use #1
print("Syrup");       //use #5
print(1999.0, 10);    //use #2
print(1999, 12);      //use #4
print(1999L, 15);     //use #3
\end{ccode}
当函数调用时未使用正确得参数类型。如:
\begin{ccode}
unsigned int year = 3210;
print(year, 6);
\end{ccode}
上述的函数调用不与任何原型匹配。此时C++将尝试使用标准类型转换强制进行匹配。如果\#2时是 print()唯一的原型,则函数调用将把 year 转换为 double 类型。但是上述“符合条件”的有三种类型。在这种情况下,C++将拒绝这种函数调用,视其为错误。

{\color{red}编译器在检查函数特征标时,将把类型引用和类型本身视为同一个特征标,以避免使用混乱。

是特征标,而不是函数类型使得可以对函数进行重载。}
\begin{ccode}
long gronk(int n, float m);
double gronk(int n, float m);
\end{ccode}
C++不允许以这种方式重载 gronk()。返回类型可以不同,但特征标也必须不同,因为是根据特征标识别调用的。
\begin{tcolorbox}[title=重载引用参数,colback=yellow!5!white,center title]
\begin{ccode}
void sink(double & r1);
void sink(const double & r2);
void sink(double && r3);
\end{ccode}
左值引用参数 r1 与可修改的左值参数匹配;const 左值引用参数 r2 与可修改的左值参数、const 左值参数和右值参数匹配;右值引用参数 r3 与右值匹配。如果重载使用这三种参数的函数,将调用最匹配的版本:
\begin{ccode}
void staff(double & rs); //matches modifiable lvalue
void staff(const double & rc); //matched rvalue,const lvalue
void stove(double & r1); //modifiable lvalue
void stove(const double & r2); //const lvalue
void stove(double && r3) //rvalue
\end{ccode}
\end{tcolorbox}
\begin{tcolorbox}[title=名片修饰,center title,colback=yellow!5!white]
C++给重载函数指定了秘密身份。使用C++开发工具中的编辑器编写和编译程序时,C++编译器将执行一些神奇的操作——名称修饰或名称矫正,它根据函数原型中指定的形参类型对每个函数名进行加密。如下述未经修饰的函数原型:\\
long MyFunctionFoo(int, float);\\
这种格式对于人类来说很合适;而编译器将名称转换为不太好看的内部表示,来描述该接口,如下所示:\\
?MyFunctionFoo@@YAXH\\
对原始名称进行的表面看来无意义的修饰将对参数数目和类型进行编码。添加的一组符号随函数特征标而异,而修饰时使用的约定随编译器而异。
\end{tcolorbox}
\section{函数模板}
函数模板是通用的函数描述,它们使用泛型来定义函数,其中的泛型可用具体的类型替换。通过将类型作为参数传递给模板,可使编译器生成该类型的函数。由于模板允许以泛型的方式编写程序,因此有时也被称为通用编程。由于类型是用参数表示的,因此模板特性有时也被称为参数化类型。

函数模板允许以任意类型的方式来定义函数。例如,可以这样建立一个交换模板:
\begin{ccode}
template <typename AnyType>
void swap(AnyType & a, AnyType & b)
{
	AnyType temp;
	temp = a;
	a = b;
	b = temp;
}
\end{ccode}
第一行指出要建立一个模板,并将其类型命名为 AnyType。关键字 template 和 typename 是必需的。在 C++98 添加关键字 typename 之前,C++使用关键字 class 来创建模板。另外尖括号也是必须的。类型名可以任意选择,只要遵守 C++命名规则即可。模板并不创建任何函数,只是告诉编译器如何定义函数。如上诉代码,需要交换 int 的函数时,编译器将按模板模式创建这样的函数,并用 int 代替 AnyType。
\begin{marker}
函数模板不能缩短可执行程序。在程序执行时,最终仍由多个独立的函数定义,就像以手工的方式定义了这些函数一样。最终的代码不包含任何模板,而只包含为程序生成的实际函数。使用模板的好处是,它使生成多个函数定义更简单,更可靠。
\end{marker}
\subsection{重载的模板及模板的局限性}
可以像重载常规函数定义那样重载模板定义,而且和常规函数重载一样,被重载的模板的函数特征标必须不同。

不过编写的模板函数很可能无法处理某些类型。即如果 T 为数组、指针或结构,T 类型的变量就不能直接进行乘除运算。