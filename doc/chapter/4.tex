\chapter{函数——C++的编程模块}
\section{函数基础知识}
要使用 C++函数,必须要完成如下工作:
\begin{itemize}
	\setlength{\itemsep}{0pt}
	\setlength{\parsep}{0pt}
	\setlength{\parskip}{0pt}
	\item 提供函数定义
	\item 提供函数原型
	\item 调用函数
\end{itemize}
\subsection{定义函数}
可以将函数分为两类:没有返回值的函数和有返回值的函数。没有返回值的函数被称为 void 函数,其通用格式为:
\begin{ccode}
void functionName (parameterList)
{
	statements;
	return;  //optional
}
\end{ccode}
有返回值的函数将生成一个值,并返回给调用函数。通用格式为:
\begin{ccode}
typeName functionName (parameterList)
{
	statements;
	return value;
}
\end{ccode}
有返回值的函数,必须使用返回语句,以便将值返回给调用函数。值本身可以是常量、变量,也可以是表达式,只是其结果的类型必须为 typeName 类型或可以被转换为 typeName(例如,如果声明是返回类型为 double,而函数返回一个 int 表达式,则该 int 值将被强制转换为 double 类型)。C++对于返回值的类型有一定的限制:不能是数组,但是可以是其它任何类型——整数、浮点数、指针,甚至可以是结构和对象!
\subsection{函数原型和函数调用}
\begin{ccode}
//using prototype and function calls
#include<iostream>
void cheers(int); //prototype;no return value
double cube(double x); //prototype; return a double
int main()
{
	...
} 
void cheers(int n)
{
	...
}
double cube(double x)
{
	return x*x*x;
}
\end{ccode}
\begin{enumerate}
	\item 为什么需要函数原型
	
	原型描述了函数到编译器的接口,它将函数返回值的类型(如果有的话)以及参数的类型和数量告诉编译器。
	\item 原型的语法
	
	函数原型是一条语句,必须以分号结束,如上述代码所示,对于 cheer()的原型,只提供了参数类型。通常在原型的参数列表中,可以包括变量名,也可以不包括变量名。原型中的变量名相当于占位符,因此不必与函数定义中的变量名相同。
\end{enumerate}
\section{函数参数和按值传递}
\subsection{数组和函数}
将数组做为函数参数时,我们知道如下恒等式
\begin{ccode}
arr == &arr[0];
arr[i] == *(arr + i);
&arr[i] == arr + i;
\end{ccode}
传递常规变量时,函数将使用该变量的拷贝;但传递数组时,由于传递的是数组的地址,函数将使用原来的数组。为防止函数无意中修改数组的内容,可在声明形参时使用关键字 const,如
\begin{ccode}
void show_array(const double ar[], int n);
\end{ccode}
该声明表明,指针 ar 指向的时常量数据。这意味着不能使用 ar 修改该数据,可以使用其值。这并不意味着原始数组必须是常量,而只是意味着不能在 show\_array()函数中使用 ar 来修改这些数据,因此,此函数将数组视为只读数据。
\subsection{指针和 const}
可以用不同的方式将 const 关键字用于指针,第一种方法是让指针指向一个常量对象,这样可以防止使用该指针来修改所指向的值,第二种方法是将指针本身声明为常量,这样可以防止改变指针指向的位置。
\begin{ccode}
//声明一个指向常量的指针
int age = 13;
const int * pt = &age;
\end{ccode}
这里将常规变量的地址赋给指向 const 的指针。还有两种可能:将 const 的变量的地址赋给指向 const 的指针(可行)、将 const 的地址赋给常规指针(不可行,表面上可以修改指针指向的值来修改变量的值,这使得解释变得模糊)
\begin{marker}
如果数据类型本身并不是指针,则可以将 const 数据或非 const 数据的地址赋给指向 const 的指针,但非 const 指针只能指向非 const 数据的地址。
\end{marker}
\begin{tcolorbox}[title=尽可能使用 const,colback=yellow!5!white,center title]
将指针参数声明为指向常量数据的指针有两条理由:
\begin{itemize}
	\setlength{\itemsep}{0pt}
	\setlength{\parsep}{0pt}
	\setlength{\parskip}{0pt}
	\item 这样可以避免由于无意间修改数据而导致的编程错误;
	\item 使用 const 使得函数能够处理 const 和非 const 实参,否则将只能接受非 const 数据。
\end{itemize}
如果条件允许,则应将指针形参声明为指向 const 的指针。
\end{tcolorbox}
第二种使用 const 的方式使得无法修改指针的值:
\begin{ccode}
//first method
int age = 23;
const int * pt = &age;
int sage = 32;
pt = &sage; //okay to point to another location
//second method
int sloth = 3;
const int * ps = &sloth; //a pointer to const int
int * const finger = &sloth; //a const pointer to int
\end{ccode}
第二种方法中,这种声明格式使得 finger 只能指向 sloth,但允许使用 finger 来修改 sloth 的值。而不能使用 ps 来修改 sloth 的值,但允许将 ps 指向另一个位置。即 finger 和*ps 都是 const,而*finger 和 ps 不是。
\section{函数和二维数组}
\section{函数和 C-风格字符串}
\section{函数和结构}
使用结构编程时,最直接的方式是像处理基本类型那样来处理结构;将结构作为参数传递,并在需要时将结构用作返回值使用。但如果结构非常大,则复制结构将增大内存要求,降低系统运行的速度。出于这些原因,许多程序员倾向于传递结构的地址,然后使用指针来访问结构的内容。
\subsection{传递方式}
\begin{itemize}
	\setlength{\itemsep}{0pt}
	\setlength{\parsep}{0pt}
	\setlength{\parskip}{0pt}
	\item 传递和返回结构
	
	当结构比较小时,按值传递结构最合理
\begin{ccode}
struct travel_time
{
	int hours;
	int mins;		
};
travel_time sum(travel_time t1, travel_time t2);
\end{ccode}
	\item 传递结构的地址
	
	假设要传递结构的地址而不是整个结构以节省时间和空间,使用指向结构的指针。
\begin{ccode}
travel_time sum(travel_time * t1, travel_time * t2);
\end{ccode}
\end{itemize}
\section{函数与对象}
\section{函数指针}
与数据项相似,函数也有地址。函数的地址是存储其机器语言代码的内存的开始地址。可以编写将另一个函数地址作为参数的函数。这样第一个函数将能够找到第二个函数,并运行它。与直接调用另一个函数相比,这种方法很笨拙。
\subsection{函数指针的基础知识}
\begin{enumerate}
	\item 获取函数的地址
	
	只要使用函数名即可。即如果 think()是一个函数,则 think 就是该函数的地址。
	\item 声明函数指针
	
	声明指向函数的指针时,必须指定指针指向的函数类型。即声明应指定函数的返回类型以及函数的特征标(参数列表)。
\begin{ccode}
double pam(int);
double (*pf)(int); //pf points to a function that takes 
//one int arg and return type double
pf = pam; //pf points to pam() function
double *pf(int); //pf() a function that return a pointer-to-double
double ned(double); 
int ted(int);
pf = ned; //invalid -- mismatched signatures
pf = ted; //invalid -- mismatched return types
\end{ccode}
	\item 使用指针来调用函数
	
	即使用指针来调用被指向的函数,如(*pf)(5)。
\end{enumerate}
\begin{ccode}
const double *(*(*pd)[3])(const double *,int) = &pa;
//pd = &pa;
\end{ccode}
如上代码所示,对于指针的初始化,是对指针变量的初始化,而不是对指针形式的初始化。即上述代码是 {\color{red}pd=\&pa},而非{\color{red}*(*(*pd)[3]=\&pa)}。即初始化的是指针,而不是指针指向的值。